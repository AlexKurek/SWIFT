\documentclass[fleqn, usenatbib, useAMS, a4paper]{mnras}
\usepackage{graphicx}
\usepackage{amsmath,paralist,xcolor,xspace,amssymb}
\usepackage{times}
\usepackage{comment}
\usepackage[super]{nth}

\newcommand{\todo}[1]{{\textcolor{red}{TODO: #1}\\}}

\newcommand{\D}[2]{\frac{d#1}{d#2}}
\newcommand{\LL}{\left(}
\newcommand{\RR}{\right)}

\title{Integration scheme for cooling}
\author{Alexei Borissov, Matthieu Schaller}

\begin{document}

\maketitle

%%%%%%%%%%%%%%%%%%%%%%%%%%%%%%%%%%%%%%%%%%%%%%%%%%%%%%%%%%%%%%%%%%%%%%%%%%%%
\section{Basic principles}

\subsection{Isochoric cooling}

\todo{Discuss the fact that we want to do cooling at constant
  density.}

\subsection{Time integration}

We want to compute the change in internal energy of a given particle
due to the interaction of the gas with the background radiation. More
specifically we want to integrate the following equation:
\begin{equation}
  u_{\rm final} \equiv u(t+\Delta t) = u(t) + \left(\frac{\partial u}{\partial t}\bigg|_{\rm
    hydro} + \frac{\partial u}{\partial t}\bigg|_{\rm cooling}\right)
  \times \Delta t.
\end{equation}
The first derivative is given by the SPH scheme, the second one is
what we are after here. We start by computing the internal energy the
particle would have in the absence of cooling:
\begin{equation}
  u_0 \equiv u(t) + \frac{\partial u}{\partial t}\bigg|_{\rm
    hydro} \times \Delta t.
\end{equation}
We then proceed to solve the implicit equation
\begin{equation}
 u_{\rm final} = u_0 + \lambda(u_{\rm final}) \Delta t,
\end{equation}
where $\lambda$ is the cooling rate, which for a given particle varies
only with respect to $u$ throughout the duration of the timestep. The
other dependencies of $\lambda$ (density, metallicity and redshift)
are kept constant over the course of $\Delta t$. Crucially, we want to
evaluate $\lambda$ at the end of the time-step. Once a solution to this
implicit problem has been found, we get the total cooling rate:
\begin{equation}
  \frac{\partial u}{\partial t}\bigg|_{\rm total} \equiv \frac{u_{\rm final} -
    u(t)}{\Delta t},
\end{equation}
leading to the following total equation of motion for internal energy:
\begin{equation}
  u(t+\Delta t) = u(t) + \frac{\partial u}{\partial t}\bigg|_{\rm
    total} \times \Delta t.
\end{equation}
The time integration is then performed in the regular time integration
section of the code. Note that, as expected, if $\lambda=0$ the whole
processes reduces to a normal hydro-dynamics only time integration of
the internal energy.

Finally, for schemes evolving entropy $A$ instead of internal energy
$u$ (or for that matter any other thermodynamic quantity), we convert
the entropy derivative coming from the hydro scheme to an internal
energy derivative, solve the implicit cooling problem using internal
energies and convert the total time derivative back to an entropy
derivative. Since we already assume that cooling is performed at
constant density, there is no loss in accuracy happening via this
conversion procedure.

%%%%%%%%%%%%%%%%%%%%%%%%%%%%%%%%%%%%%%%%%%%%%%%%%%%%%%%%%%%%%%%%%%%%%%%%%%%%
\section{Solution to the implicit cooling problem}

\subsection{Explicit solver}

\todo{Explain the explicit solver}

\subsection{Newton-Raphson method}

To prevent the occurance of negative internal energies during the
calculation we introduce $x = \log u$, so that we need to solve
\begin{equation}\label{fx-eq}
f(x) = e^x - u_0 - \lambda(e^x) dt = 0.
\end{equation}
Using Newton's method we obtain consecutive approximations of the root
of $f$ by the formula $x_{n+1} = x_n - f(x_n)/f'(x_n)$. This leads to
\begin{equation}
x_{n+1} = x_n - \frac{1 - u_0 e^{-x_n} -\lambda(e^{x_n})e^{-x_n}dt}{1
  - \frac{d\lambda}{du}(e^{x_n}) dt}.
\end{equation}
We obtain the gradient by
\begin{equation}
  \D \lambda u = \frac{\lambda(u_{high,n})
    - \lambda(u_{low,n})}{u_{high,n} - u_{low,n}},
\end{equation}
where $u_{\rm high,n}$ and $ u_{\rm low,n}$ are values of the internal
energy grid bracketing the current iteration of the value of the
internal energy ($u_n = e^{x_n}$) in Newton's method (i.e. $u_{high,n}
\ge u_n \ge u_{low,n}$).

%\begin{figure} \begin{center} \includegraphics[width =
%0.7\textwidth]{typical_fx} \caption{Relationship of $\log|f(x)|$ from
%Equation \ref{fx-eq} to the logarithm of temperature for multiple
%values $u_0$. Solid lines indicate where the value of $f(x)$ is
%negative, while for dashed lines $f(x)$ is positive. A hydrogen
%number density of $10^{-1}$ cm$^{-1}$, solar abundances and redshift
%0 are used for evaluating the cooling
%rate.}  \label{fx} \end{center} \end{figure}

The root of $f$ tends to be near the location of maximum gradient, so
the initial guess for the Newton's method is chosen to be the location
where $g(x) = e^x - u_0 - \lambda_h(e^x) dt$ has the greatest slope,
with $\lambda_h$ being the contribution to the cooling from hydrogen
and helium. Since the cooling rate is dominated by hydrogen and helium
at lower temperatures, this is a suitable way of approximating the
equilibrium temperature, and supplying a guess to the Newton iteration
scheme.

A particle is considered to have converged if the relative error in
the internal energy is sufficiently small. This can be formulated as
\begin{align*}
\frac{u_{n+1} - u_n}{u_{n+1}} &< C \\
u_{n+1} - u_n &< Cu_{n+1} \\
\LL 1-C\RR u_{n+1} &< u_n \\
\frac{u_{n+1}}{u_n} &< \frac{1}{1-C} \\
x_{n+1} - x_n = \log\frac{u_{n+1}}{u_n} &< -\log\LL 1-C \RR \simeq C.
\end{align*}
Since the grid spacing in the internal energy of the Eagle tables is
0.045 in $\log_{10}u$ we take $C = 10^{-2}$.

\subsection{Bissection method}

\todo{Explain the bissection method}

\section{EAGLE cooling tables}

\todo{Summarize the content of the Wiersma tables.}
\todo{Summarize the high-redshift tables.}
\todo{Explain the Compton-cooling contribution.}

\section{Co-moving time integration}

In the case of cosmological simulations, the equations need to be
slightly modified to take into account the expansion of the
Universe. The code uses the comoving internal energy $u' =
a(t)^{3(\gamma-1)}u$ or comoving entropy $A'=A$ as thermodynamic
variable. The equation of motion for the variable are then modified
and take the following form:
\begin{equation}
  \frac{\partial u'_i}{\partial t} = \frac{\partial u'_i}{\partial
    t}\bigg|_{\rm hydro}  = \frac{1}{a(t)^2} Y'_i(t)\big|_{\rm
    hydro},
\end{equation}
where $Y_i$ is computed from the particle itself and its neighbours
and corresponds to the change in internal energy due to hydrodynamic
forces. We then integrate the internal energy forward in time using
\begin{equation}
  u'_i(t+\Delta t) = u'_i(t) + Y'_i(t)\big|_{\rm hydro} \times \underbrace{\int_t^{t+\Delta t}
  \frac{1}{a(t)^2} dt}_{\Delta t_{\rm therm}}.
\end{equation}
The exact same equations apply in the case of a hydrodynamics scheme
evolving entropy (see cosmology document). We note that this is
different from the choice made in Gadget where there is no $a^{-2}$
term as it is absorbed in the definition in $Y'_i$ itself. As a
consequence $\Delta t_{\rm therm}$ is just $\Delta t$.

In order to compute the
cooling rate of a particle, we convert quantities to physcial
coordinates. Given the appearence of scale-factors in some of these
equations, we have to be careful to remain consistent throughout. We
start by constructing the co-moving internal energy at the end of the
time-step in the absence of cooling:
\begin{equation}
  u'_0 \equiv u'(t) + Y'_i(t)\big|_{\rm hydro} \times \Delta t_{\rm therm},
\end{equation}
which we then convert into a physical internal energy alongside the
thermal energy at the current time:
\begin{align}
  u(t) &= a^{3(1-\gamma)}u'(t),\\
  u_0 &= a^{3(1-\gamma)}u'_0.
\end{align}
We can then solve the implicit cooling problem in the same way as in
the non-comoving case and obtain
\begin{equation}
  u_{\rm final} = u_0 + \lambda(u_{\rm final}) \Delta t.
\end{equation}
We note that the $\Delta t$ here is the actual time between the start
and end of the step; unlike $\Delta t_{\rm therm}$ there are no
scale-factors entering that term. The solution to the implicit problem
in physical coordinates yields the definition of the total time
derivative of internal energy:
\begin{equation}
  \frac{\partial u}{\partial t}\bigg|_{\rm total} \equiv \frac{u_{\rm final} -
    u(t)}{\Delta t}.
\end{equation}
This allows us to construct the total eveolution of co-moving energy
is
\begin{equation}
  Y'_i(t)\big|_{\rm total} = a^2\times a^{3(\gamma-1)} \times
  \frac{\partial u}{\partial t}\bigg|_{\rm total},
\end{equation}
where the first term is requred our definition of $\Delta t_{\rm
  therm}$ and the second one is the conversion from physical to
co-moving internal energy. The time integration routine then performs
the same calculation as in the non-cooling case:
\begin{equation}
  u'_i(t+\Delta t) = u'_i(t) + Y'_i(t)\big|_{\rm total} \times {\Delta t_{\rm therm}}.
\end{equation}
\end{document}
