\subsection{Choice of co-moving coordinates}
\label{ssec:ccordinates}

Note that, unlike the convention used by many codes, we do not express
quantities with ``little h'' ($h$) included; for instance units of
length are expressed in units of $\rm{Mpc}$ and not
${\rm{Mpc}}/h$. Similarly, the time integration operators (see
below)include an $h$-factor via the explicit appearance of
the Hubble constant.\\

In physical coordinates, the Lagrangian for a particle $i$ in the
\cite{Springel2002} flavour of SPH with gravity reads
\begin{equation}
  \Lag =
  \frac{1}{2} m_i \dot{\mathbf{r}}_i^2 -
  \frac{1}{\gamma-1}m_iA_i\rho_i^{\gamma-1} -
  m_i \phi
\end{equation}
Introducing the comoving positions $\mathbf{r}'$ such that $\mathbf{r}
= a(t) \mathbf{r}'$ and comoving densities $\rho' \equiv a^3(t)\rho$,
\begin{equation}
  \Lag =
  \frac{1}{2} m_i \left(a\dot{\mathbf{r}}_i' + \dot{a}\mathbf{r}_i'
  \right)^2 - 
  \frac{1}{\gamma-1}m_iA_i'\left(\frac{\rho_i'}{a^3}\right)^{\gamma-1}
  - m_i \phi,
\end{equation}
where $A'=A$ is chosen such that the equation of state for
the gas and thermodynamic relations between quantities have the same
form (i.e. are scale-factor free) in the primed coordinates as
well. This implies
\begin{equation}
  P' = a^{3\gamma}P,\quad u'=a^{3(\gamma-1)}u, \quad c'=a^{3(\gamma-1)/2}c,
\end{equation}
for the pressure, internal energy and sound-speed
respectively. Following \cite{Peebles1980} (ch.7), we introduce the
gauge transformation $\Lag \rightarrow \Lag + \frac{d}{dt}\Psi$ with
$\Psi \equiv \frac{1}{2}a\dot{a}\mathbf{r}_i^2$ and obtain
\begin{align}
  \Lag &= \frac{1}{2}m_ia^2 \dot{\mathbf{r}}_i'^2 -
  \frac{1}{\gamma-1}m_iA_i'\left(\frac{\rho_i'}{a^3}\right)^{\gamma-1}
  -\frac{\phi'}{a},\\
  \phi' &= a\phi + \frac{1}{2}a^2\ddot{a}\mathbf{r}_i'^2,\nonumber
\end{align}
and call $\phi'$ the peculiar potential.  Finally, we introduce the
velocities used internally by the code:
\begin{equation}
  \mathbf{v}' \equiv a^2\dot{\mathbf{r}'}.
\end{equation}
Note that these velocities \emph{do not} have a physical
interpretation. We caution that they are not the peculiar velocities
($\mathbf{v}_{\rm p} \equiv a\dot{\mathbf{r}'} =
\frac{1}{a}\mathbf{v}'$), nor the Hubble flow
($\mathbf{v}_{\rm H} \equiv \dot{a}\mathbf{r}'$), nor the total
velocities
($\mathbf{v}_{\rm tot} \equiv \mathbf{v}_{\rm p} + \mathbf{v}_{\rm H}
= \dot{a}\mathbf{r}' + \frac{1}{a}\mathbf{v}'$) and also differ from
the convention used in \gadget snapshots
($\sqrt{a} \dot{\mathbf{r}'}$) and other related simulation
codes\footnote{One additional inconvenience of our choice of
  generalised coordinates is that our velocities $\mathbf{v}'$ and
  sound-speed $c'$ do not have the same dependencies on the
  scale-factor. The signal velocity entering the time-step calculation
  will hence read
  $v_{\rm sig} = a\dot{\mathbf{r}'} + c = \frac{1}{a} \left(
    |\mathbf{v}'| + a^{(5 - 3\gamma)/2}c'\right)$.}.  For the
derivatives, this choice implies
$\dot{\mathbf{v}'} = 2H\mathbf{v}' + a^2\ddot{\mathbf{r}'}$.


