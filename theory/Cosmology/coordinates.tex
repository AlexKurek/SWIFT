\subsection{Choice of co-moving coordinates}
\label{ssec:ccordinates}

Note that unlike the \gadget convention we do not express quantities
with ``little h'' included. We hence use $\rm{Mpc}$ and not
${\rm{Mpc}}/h$ for instance. Similarly, the time integration operators
(see below) also include an $h$-factor via the explicit appearance of
the Hubble constant.\\
In physical coordinates, the Lagrangian for a particle $i$ in the
\cite{Springel2002} flavour of SPH with gravity reads
\begin{equation}
  \Lag =
  \frac{1}{2} m_i \dot{\mathbf{r}}_i^2 -
  \frac{1}{\gamma-1}m_iA_i\rho_i^{\gamma-1} -
  m_i \phi
\end{equation}
Introducing the comoving positions $\mathbf{r}'$ such that $\mathbf{r}
= a(t) \mathbf{r}'$ and comoving densities $\rho' \equiv a^3(t)\rho$ ,we get
\begin{equation}
  \Lag =
  \frac{1}{2} m_i \left(a\dot{\mathbf{r}}_i' + \dot{a}\mathbf{r}'
  \right)^2 - 
  \frac{1}{\gamma-1}m_iA_i'\left(\frac{\rho_i'}{a^3}\right)^{\gamma-1}
  - m_i \phi,
\end{equation}
where we chose to define $A'=A$ such that the equation of state for
the gas and thermodynamics relations between quantites have the same
form (i.e. are scale-factor free) in the primed coordinates as
well. This implies
\begin{equation}
  P' = a^{3\gamma}P,\quad u'=a^{3(\gamma-1)}u, \quad c'=a^{3(\gamma-1)/2}c,
\end{equation}
for the pressure, internal energy and sound-speed
respectively. Following \cite{Peebles1980} (ch.7), we introduce the
gauge transformation $\Lag \rightarrow \Lag + \frac{d}{dt}\Psi$ with
$\Psi \equiv \frac{1}{2}a\dot{a}\mathbf{r}_i^2$ and obtain
\begin{align}
  \Lag &= \frac{1}{2}m_ia^2 \dot{\mathbf{r}}_i^2 -
  \frac{1}{\gamma-1}m_iA_i'\left(\frac{\rho_i'}{a^3}\right)^{\gamma-1}
  -\frac{\phi'}{a},\\
  \phi' &= a\phi + \frac{1}{2}a^2\ddot{a}\mathbf{r}_i'^2.\nonumber
\end{align}
Finally, we introduce the velocities $\mathbf{v}' \equiv
a^2\dot{\mathbf{r}'}$ used internally by the code. Note that these
velocites do not have a physical interpretation. We caution that they
are not the peculiar velocities, nor the Hubble flow, nor the total
velocities\footnote{One additional inconvenience of our choice of
  generalized coordinates is that our velocities $\mathbf{v}'$ and
  sound-speed $c'$ do not have the same dependencies on the
  scale-factor. The signal velocity entering the time-step calculation
  will hence read $v_{\rm sig} = a\dot{\mathbf{r}'} + c =
  \frac{|\mathbf{v}'|}{a} + a^{1-3(\gamma-1)/2}c'$.}. Using the SPH
definition of density, $\rho_i =
\sum_jm_jW(\mathbf{r}_{j}'-\mathbf{r}_{i}',h_i') =
\sum_jm_jW_{ij}'(h_i')$, we can follow \cite{Price2012} andapply the
Euler-Lagrange equations to write
\begin{alignat}{3}
  \dot{\mathbf{r}}_i'&= \frac{1}{a^2} \mathbf{v}_i'&  \label{eq:cosmo_eom_r} \\
  \dot{\mathbf{v}}_i' &= \sum_j m_j &&\left[\frac{1}{a^{3(\gamma-1)}}f_i'A_i'\rho_i'^{\gamma-2}\mathbf{\nabla}_i'W_{ij}'(h_i)\right. \nonumber\\
  &   && + \left. \frac{1}{a^{3(\gamma-1)}}f_j'A_j'\rho_j'^{\gamma-2}\mathbf{\nabla}_i'W_{ij}'(h_j)\right. \nonumber\\
  &   && + \left. \frac{1}{a}\mathbf{\nabla}_i'\phi'\right] \label{eq:cosmo_eom_v}
\end{alignat}
with
\begin{equation}
    f_i' = \left[1 + \frac{h_i'}{3\rho_i'}\frac{\partial
      \rho_i'}{\partial h_i'}\right]^{-1}, \qquad \mathbf{\nabla}_i'
  \equiv \frac{\partial}{\partial \mathbf{r}_{i}'}. \nonumber
\end{equation}
These corresponds to the equations of motion for density-entropy SPH
\citep[e.g. eq. 14 of][]{Hopkins2013} with cosmological and
gravitational terms. SPH flavours that evolve the internal energy $u$ instead of the
entropy the additional equation of motion describing the evolution of
$u'$ becomes:
\begin{equation}
  \dot{u}_i' = \frac{P_i'}{\rho_i'^2}\left[3H\rho_i' + \frac{1}{a^2}f_i'\sum_jm_j\left(\mathbf{v}_i' -
  \mathbf{v}_j'\right)\mathbf{\nabla}_i'W_{ij}'(h_i)\right],
\end{equation}
where the first term in the brackets accounts for the change in energy
due to the expansion of the Universe. The scale-factors appearing in
the equations are later absorbed in the time-integration operators
(Sec.~\ref{ssec:operators}) such that the RHS of the equations of
motions is identical for the primed quantities to the ones obtained in
the non-cosmological case for the physical quantities.

Additional terms in the SPH equations of motion (e.g. viscosity
switches) often rely on the velocity divergence and curl. Their SPH
estimators $\langle\cdot\rangle$ in physical coordinates can be
related to their estimators based on our primed-coordinates using:
\begin{align}
  \left\langle \mathbf{\nabla}\cdot\dot{\mathbf{r}}_i \right\rangle &=
  \frac{1}{a^2} \left\langle
  \mathbf{\nabla}'\cdot\mathbf{v}_i'\right\rangle =
  \frac{1}{a^2\rho_i'}\sum_j m_j\left(\mathbf{v}_j' -
  \mathbf{v}_i'\right) \cdot \mathbf{\nabla}_i'W_{ij}'(h_i), \nonumber \\
  \left\langle \mathbf{\nabla}\times\dot{\mathbf{r}}_i \right\rangle &=
  \frac{1}{a^2} \left\langle
  \mathbf{\nabla}'\times\mathbf{v}_i'\right\rangle =
  \frac{1}{a^2\rho_i'}\sum_j m_j\left(\mathbf{v}_j' -
  \mathbf{v}_i'\right) \times \mathbf{\nabla}_i'W_{ij}'(h_i) \nonumber
\end{align}

