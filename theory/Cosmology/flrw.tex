\subsection{Background evolution}
\label{ssec:flrw}

In \swift we assume a standard FLRW metric for the evolution of the
background density of the Universe and use the Friedmann equations to
describe the evolution of the scale-factor $a(t)$.  As always, we
scale $a$ such that its value at present-day is
$a_0 \equiv a(t=t_{\rm now}) = 1$. We also define redshift
$z \equiv 1/a - 1$ and the Hubble parameter at a given redshift
\begin{equation}
H(t) \equiv \frac{\dot{a}(t)}{a(t)}
\end{equation}
with its present-day value denoted as $H_0 = H(t=t_{\rm
  now})$. Following normal conventions we write
$H_0 = 100 h~\rm{km}\cdot\rm{s}^{-1}\cdot\rm{Mpc}^{-1}$ and use $h$ as
the input parameter for the Hubble constant.

To allow for general expansion histories we use the full Friedmann
equations and write
\begin{align}
H(a) &\equiv H_0 E(a) \\
     E(a) &\equiv\sqrt{\Omega_m a^{-3} + \Omega_r a^{-4} + \Omega_k a^{-2} + \Omega_\Lambda a^{-3(1+w(a))}},
\label{eq:friedmann}
\end{align}
where the dark energy equation-of-state is evolving according to the
formulation of \cite{Linder2003}:
\begin{equation}
w(a) \equiv w_0 + w_a(1-a).
\end{equation}
The cosmological model is hence fully defined by specifying the
dimensionless constants $\Omega_m$, $\Omega_r$, $\Omega_k$,
$\Omega_\Lambda$, $h$, $w_0$ and $w_a$ as well as the starting
redshift (or scale-factor of the simulation) $a_{\rm start}$ and final
time $a_{\rm end}$. \\ At any scale-factor $a_{\rm age}$, the time
$t_{\rm age}$ since the Big Bang (age of the Universe) can be computed
as \citep[e.g.][]{Wright2006}:
\begin{equation}
  t_{\rm age} = \int_{0}^{a_{\rm age}} dt = \int_{0}^{a_{\rm age}} \frac{da}{a \cdot H(a)} = \frac{1}{H_0} \int_{0}^{a_{\rm age}} \frac{da}{a \cdot E(a)}.
\end{equation}
For a general set of cosmological parameters, this integral can only
be evaluated numerically, which is too slow to be evaluated accurately
during a run. At the start of the simulation we, hence, evaluate this
integral for $10^4$ values of $a_{\rm age}$ equally spaced between
$\log(a_{\rm start})$ and $\log(a_{\rm end})$. These are obtained via
adaptive quadrature using the 61-points Gauss-Konrod rule implemented
in the {\sc gsl} library \citep{GSL} with a relative error limit of
$\epsilon=10^{-10}$. The value for a specific $a$ (during an actual
run) is then obtained by linear interpolation of that table.

