\subsection{Derviation of the GIZMO equations}

For GIZMO and other finite volume/mass schemes, the equations of motion can no
longer be derived from a Lagrangian. Instead, we have to explicitly transform
the Euler equations to comoving coordinates and variables. In a physical
coordinate frame, these equations are
\begin{align}
\frac{\partial{}\rho{}}{\partial{}t} +
\nabla{} \cdot \left(\rho{} \mathbf{v}_{\rm tot} \right)
&= 0, \\
\frac{\partial{} \rho{} \mathbf{v}_{\rm tot} }{\partial{} t} +
\nabla{} \cdot \left( \rho{} \mathbf{v}_{\rm tot} \mathbf{v}_{\rm tot} +
P \mathbb{1} \right) &= 0.
\end{align}
\begin{multline}
\frac{\partial{}}{\partial{} t} \left( \rho{} u + \frac{1}{2} \rho{}
\mathbf{v}_{\rm{} tot}\cdot\mathbf{v}_{\rm{} tot} \right) +\\
\nabla{} \cdot \left( \rho{} u \mathbf{v}_{\rm{} tot} +
\frac{1}{2} \rho{} \left(\mathbf{v}_{\rm{} tot} \cdot \mathbf{v}_{\rm{} tot}\right)
\mathbf{v}_{\rm{} tot} + P \mathbf{v}_{\rm{} tot} \right) = 0,
\end{multline}
where $\mathbb{1}$ is the unit tensor.
For simplicity, we will rewrite the last two equations in terms of the quantities
$\mathbf{v}_{\rm{} tot}$ and $u$:
\begin{align}
\frac{\partial{} \mathbf{v}_{\rm{} tot}}{\partial{} t} +
\left(\mathbf{v}_{\rm{} tot} \cdot \nabla{} \right) \mathbf{v}_{\rm{} tot} +
\frac{1}{\rho{}}\nabla{} P &= 0,\\
\frac{\partial{} u}{\partial{} t} +
\left(\mathbf{v}_{\rm{} tot} \cdot \nabla{} \right) u +
\frac{P}{\rho{}} \nabla{} \cdot \mathbf{v}_{\rm{} tot} &= 0.
\end{align}
To convert to comoving coordinates, we need to take into account the
appropriate operator transformations:
\begin{align}
\nabla{} &\rightarrow{} \frac{1}{a} \nabla{}', \\
\frac{\partial{}}{\partial{} t} &\rightarrow{}
\frac{\partial{}}{\partial{} t} - \frac{\dot{a}}{a} \mathbf{r}' \cdot \nabla{}',
\end{align}
the latter of which follows from the explicit time dependence of the
comoving coordinate $\mathbf{r}'$. Substituting the definitions of the
comoving variables and operators into the first Euler equation, we
obtain
\begin{equation}
\frac{\partial{} \rho{}'}{\partial{} t} + \frac{1}{a^2} \nabla{}' \cdot \left(
\rho{}' \mathbf{v}' \right) = 0,
\end{equation}
which is the same as the original continuity equation but now with an extra
$1/a^2$ for the second term (the same correction factor that appears
for the ``drift'' in the SPH case). For the velocity equation, we find
\begin{multline}
\frac{\partial{} \mathbf{v}'}{\partial{} t} + \frac{1}{a^2} \left( \mathbf{v}' \cdot
\nabla{}' \right) \mathbf{v}' + \frac{1}{a^{3(\gamma{} - 1)}\rho{}'}
\nabla{}' P' = \\
- \nabla{}' \left( \frac{1}{2} a \ddot{a} \mathbf{r}'\cdot\mathbf{r}'
\right).
\end{multline}
The right hand side of this equation is simply the extra term we absorb in the
potential through the gauge transformation; the $1/a^2$ dependence in the
second term on the left hand side again corresponds to the SPH ``drift'' factor,
while the more complicated correction factor for the third term corresponds
to the SPH ``kick'' factor in the equation of motion for the
velocity. Finally, the thermal energy equation reduces to
\begin{equation}
\frac{\partial{} u'}{\partial{} t} + \frac{1}{a^2} \left( \mathbf{v}' .
\nabla{}' \right) u' + \frac{1}{a^2} \frac{P'}{\rho{}'} \nabla{}' \cdot \mathbf{v}'
= 0.
\end{equation}
Again, this gives us the same correction factors as used by the SPH energy
equation.\\

Unfortunately, the system of equations above is no longer consistent with the
original equations. For the continuity equation, we can transform to a new
time coordinate $t'$ defined by
\begin{equation}
\dot{t}' = \frac{1}{a^2}, \quad{} t'(t) = \int \frac{dt}{a^2(t)} + {\rm{}const,}
\end{equation}
and end up with the original continuity equation in comoving variables. The
same transformation brings the thermal energy equation into its original form
in comoving variables. Unfortunately, the same does not work for the velocity
equation due to the $a^{-3(\gamma{}-1)}$ factor in the third term (note that
for the specific choice $\gamma{}=5/3$ this procedure would work).
To get around this issue, we will rewrite the velocity equation as
\begin{multline}
\frac{\partial{} \mathbf{v}'}{\partial{} t} + \frac{1}{a^2} \left(
\mathbf{v}' \cdot
\nabla{}' \right) \mathbf{v}' + \frac{1}{a^2\rho{}'}
\nabla{}' P' = \\
- \nabla{}' \left( \frac{1}{2} a \ddot{a} \mathbf{r}'\cdot\mathbf{r}'
\right) - \left(\frac{1}{a^{3(\gamma{} - 1)}} - \frac{1}{a^2} \right)
\nabla{}'P'
\end{multline}
and treat the extra correction term on the right hand side as a source term,
like we do for the gravitational acceleration. This means we end up with
a fully consistent set of comoving Euler equations, so that we can solve the
Riemann problem in the comoving frame.

If we now convert the primitive Euler equations back to their conservative form,
we end up with the following set of equations:
\begin{equation}
\frac{\partial{} \rho{}'}{\partial{} t} + \frac{1}{a^2} \nabla{}' \cdot \left(
\rho{}' \mathbf{v}' \right) = 0,
\end{equation}
\begin{multline}
\frac{\partial{} \rho{}' \mathbf{v}'}{\partial{} t} + \frac{1}{a^2}
\nabla{} \cdot \left( \rho{}' \mathbf{v}'\mathbf{v}' + P' \right) = \\
- \rho{}' \nabla{}' \left( \frac{1}{2} a \ddot{a} \mathbf{r}'\cdot\mathbf{r}'
\right) - \left(\frac{1}{a^{3(\gamma{} - 1)}} - \frac{1}{a^2} \right)
\nabla{}'P',
\end{multline}
\begin{multline}
\frac{\partial{}}{\partial{} t} \left( \rho{}' u' + \frac{1}{2} \rho{}'
\mathbf{v}'\cdot\mathbf{v}' \right) +\\
\frac{1}{a^2}\nabla{} \cdot \left( \rho{}' u' \mathbf{v}' +
\frac{1}{2} \rho{}' \left(\mathbf{v}'\cdot \mathbf{v}'\right)
\mathbf{v}' + P' \mathbf{v}' \right) =\\
- \rho{}' \mathbf{v}'\cdot\nabla{}' \left( \frac{1}{2} a \ddot{a} \mathbf{r}'\cdot
\mathbf{r}' \right) -
\left(\frac{1}{a^{3(\gamma{} - 1)}} - \frac{1}{a^2} \right)
\mathbf{v}'\cdot\nabla{}'P'.
\end{multline}
These equations tell us that the mass, comoving momentum and comoving total
energy are conserved in the case of adiabatic expansion ($\gamma{} = 5/3$).
For a more general $\gamma{}$, we however end up with extra source terms that
depend on $\nabla{}'P'$. Since we already use this quantity for the gradient
reconstruction step, adding this term is straightforward. The additional time
step required to integrate the source term is
\begin{multline}
\Delta{} t_{\rm{} kick,c} \equiv \int_{a_n}^{a_{n+1}}
\left(\frac{1}{a^{3(\gamma{} - 1)}} - \frac{1}{a^2} \right) dt \\ =
\Delta{} t_{\rm{} kick,h} - \Delta{} t_{\rm{} drift}.
\end{multline}
