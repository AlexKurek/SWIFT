\subsection{Time-integration operators}
\label{ssec:operators}
For the choice of cosmological coordinates made in \swift, the normal
``kick'' and ``drift'' operators get modified to account for the
expansion of the Universe. The derivation of these operators from the
system's Lagrangian is given in appendix A of \cite{Quinn1997} for the
collision-less case. We do not repeat that derivation here but, for
completeness, give the expressions we use as well as the ones used for
the hydro-dynamics. 

The ``drift'' operator gets modified such that $\Delta t$ for a
time-step running from a scale-factor $a_{n}$ to $a_{n+1}$ becomes

\begin{equation}
  \Delta t_{\rm drift} \equiv \int_{a_n}^{a_{n+1}} \frac{dt}{a^2} = \frac{1}{H_0} \int_{a_n}^{a_{n+1}} \frac{da}{a^3E(a)},
\end{equation}
with $E(a)$ given by eq.~\ref{eq:friedmann}. Similarly, the time-step
entering ``kick'' operator for collision-less acceleration reads
\begin{equation}
  \Delta t_{\rm kick,g} \equiv \int_{a_n}^{a_{n+1}} \frac{dt}{a} = \frac{1}{H_0} \int_{a_n}^{a_{n+1}} \frac{da}{a^2E(a)}.
\end{equation}
However, for the case of gas dynamics, given our choice of
coordinates, the ``kick'' operator has a second variant that reads
\begin{equation}
  \Delta t_{\rm kick,h} \equiv \int_{a_n}^{a_{n+1}} \frac{dt}{a^{3(\gamma-1)}} = \frac{1}{H_0} \int_{a_n}^{a_{n+1}} \frac{da}{a^{3\gamma - 2}E(a)},
\end{equation}
where $\gamma$ is the adiabatic index of the gas.  Accelerations
arising from hydrodynamic forces (\nth{1} and \nth{2} term in
eq.~\ref{eq:cosmo_eom_v}) are integrated forward in time using $\Delta
t_{\rm kick,h}$, whilst the accelerations given by the gravity forces
(\nth{3} term in eq.~\ref{eq:cosmo_eom_v}) use $\Delta t_{\rm
  kick,g}$. The entropy or internal energy is integrated forward in
time using $\Delta t_{\rm kick,A} = \Delta t_{\rm
  drift}$\footnote{Note that {\sc gadget-2} uses a slightly different
  operator here. They first multiply $\dot{A}_i'$ by $1/H$ and do not
  not consider the $1/a^2$ term as part of the time-integration
  operator. They then use $\int H dt$ as the operator, which
  integrates out trivially. This slight inconsistency with the rest of
  the time-integration operators is unlikely to lead to any practical
  difference.}. We additionally compute a few other terms
appearing in some viscosity terms and subgrid models. There are the
difference in cosmic time between the start and the end of the step
and the corresponding change in redshift:
\begin{align}
  \Delta t_{\rm cosmic} &= \int_{a_n}^{a_{n+1}} dt = \frac{1}{H_0}
  \int_{a_n}^{a_{n+1}} \frac{da}{a E(a)},\\
  \Delta z &= \frac{1}{a_n} - \frac{1}{a_{n+1}} \approx -\frac{H}{a} \Delta t_{\rm cosmic}.
\end{align}
Following the same method as for the age of the Universe
(sec. \ref{ssec:flrw}), these three non-trivial integrals are
evaluated numerically at the start of the simulation for a series
$10^4$ values of $a$ placed at regular intervals between $\log a_{\rm
  begin}$ and $\log a_{\rm end}$. The values for a specific pair of
scale-factors $a_n$ and $a_{n+1}$ are then obtained by interpolating
that table linearly.


