\subsection{Derivatives of the potential}

For completeness, we give here the full expression for the first few
derivatives of the potential that are used in our FMM scheme. We use
the notation $\mathbf{r}=(r_x, r_y, r_z)$, $r = |\mathbf{r}|$ and
$u=r/H$. Starting from the potential (Eq. \ref{eq:fmm:potential},
reproduced here for clarity), 
\begin{align}
D_{000}(\mathbf{r}) = \phi (\mathbf{r},H) = 
\left\lbrace\begin{array}{rcl}
\frac{1}{H} \left(-3u^7 + 15u^6 - 28u^5 + 21u^4 - 7u^2 + 3\right) & \mbox{if} & u < 1,\\
\frac{1}{r} & \mbox{if} & u \geq 1, 
\end{array}
\right.\nonumber
\end{align}
we can construct the higher order terms by successively applying the
"chain rule". We show examples of the first few relevant ones here.

\begin{align}
D_{100}(\mathbf{r}) = \frac{\partial}{\partial r_x} \phi (\mathbf{r},H) = 
\left\lbrace\begin{array}{rcl}
\frac{r_x}{H^3} \left(-21u^5 + 90u^4 - 140u^3 + 84u^2 - 14\right) & \mbox{if} & u < 1,\\
-\frac{r_x}{r^3} & \mbox{if} & u \geq 1, 
\end{array}
\right.\nonumber
\end{align}

\begin{align}
D_{200}(\mathbf{r}) = \frac{\partial^2}{\partial r_x^2} \phi (\mathbf{r},H) = 
\left\lbrace\begin{array}{rcl}
\frac{r_x^2}{H^5}\left(-105u^3+360u^2-420u+168\right) +
\frac{1}{H^3} \left(-21u^5 + 90u^4 - 140u^3 + 84u^2 - 14\right) & \mbox{if} & u < 1,\\
3\frac{r_x^2}{r^5} - \frac{1}{r^3} & \mbox{if} & u \geq 1, 
\end{array}
\right.\nonumber
\end{align}

\begin{align}
D_{110}(\mathbf{r}) = \frac{\partial^2}{\partial r_x\partial r_y} \phi (\mathbf{r},H) = 
\left\lbrace\begin{array}{rcl}
\frac{r_xr_y}{H^5}\left(-105u^3+360u^2-420u+168\right) & \mbox{if} & u < 1,\\
3\frac{r_xr_y}{r^5} & \mbox{if} & u \geq 1, 
\end{array}
\right.\nonumber
\end{align}
