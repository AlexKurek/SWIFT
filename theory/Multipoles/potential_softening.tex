\subsection{Gravitational softening}
\label{ssec:potential_softening}

To avoid artificial two-body relaxation, the Dirac
$\delta$-distribution of particles is convolved with a softening
kernel of a given fixed, but time-variable, scale-length
$\epsilon$. Instead of the commonly used spline kernel of
\cite{Monaghan1985} (e.g. in \textsc{Gadget}), we use a C2 kernel
\citep{Wendland1995} which leads to an expression for the force that
is cheaper to compute and has a very similar overall shape. The C2
kernel has the advantage of being branch-free leading to an expression
which is faster to evaluate using vector units available on modern
architectures; it also does not require any divisions to evaluate the
softened forces. We set $\tilde\delta(\mathbf{x}) =
\rho(|\mathbf{x}|) = W(|\mathbf{x}|, 3\epsilon_{\rm Plummer})$, with
$W(r, H)$ given by

\begin{align}
W(r,H) &= \frac{21}{2\pi H^3} \times \nonumber \\
&\left\lbrace\begin{array}{rcl}
4u^5 - 15u^4 + 20u^3 - 10u^2 + 1 & \mbox{if} & u < 1,\\
0 & \mbox{if} & u \geq 1,
\end{array}
\right.
\end{align}
and $u = r/H$. The potential $\varphi(r,H)$ corresponding to this density distribution reads
\begin{align}
\varphi = 
\left\lbrace\begin{array}{rcl}
\frac{1}{H} (-3u^7 + 15u^6 - 28u^5 + 21u^4 - 7u^2 + 3) & \mbox{if} & u < 1,\\
\frac{1}{r} & \mbox{if} & u \geq 1.
\end{array}
\right.
\label{eq:fmm:potential}
\end{align}

These choices, lead to a potential at $|\mathbf{x}| = 0$ equal to the
central potential of a Plummer sphere (i.e. $1/\epsilon_{\rm
Plummer}$)\footnote{Note the factor $3$ in the definition of
$\rho(|\mathbf{x}|)$ which differs from the factor $2.8$ used
in \textsc{Gadget} as a consequence of the change of kernel
shape.}. The softened density profile, its corresponding potential and
resulting forces are shown on Fig. \ref{fig:fmm:softening} (for
details see Sec. 2 of~\cite{Price2007}).


\begin{figure}
\includegraphics[width=\columnwidth]{potential.pdf}
\caption{The density (top), potential (middle) and forces (bottom)
  generated py a point mass in our softened gravitational scheme.
  A Plummer-equivalent sphere is shown for comparison. The spline
  kernel of \citet{Monaghan1985}, used in \textsc{Gadget}, is shown
  for comparison but note that it has not been re-scaled to match the
  Plummer-sphere potential at $r=0$.  %(for completeness, we chose
  %$\epsilon=2$).
  }
\label{fig:fmm:softening}
\end{figure}
