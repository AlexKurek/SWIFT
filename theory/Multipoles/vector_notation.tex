\section{Multi-index notation}
\label{sec:multi_index_notation}

We define a multi-index $\mathbf{n}$ as a triplet of integers
non-negative integers:
\begin{equation}
  \mathbf{n} \equiv \left(n_x, n_y, n_z\right), \qquad n_i \in \mathbb{N},
\end{equation}
with a norm $n$ given by
\begin{equation}
  n = |\mathbf{n}| \equiv n_x + n_y + n_z. 
\end{equation}
We also define the exponentiation of a vector
$\mathbf{r}=(r_x,r_y,r_z)$ by a multi-index $\mathbf{n}$ as
\begin{equation}
  \mathbf{r}^\mathbf{n} \equiv r_x^{n_x} \cdot r_y^{n_y} \cdot r_z^{n_z},
\end{equation}
which for a scalar $\alpha$ reduces to
\begin{equation}
  \alpha^\mathbf{n} = \alpha^{n}.
\end{equation}
Finally, the factiorial of a multi-index is defined to be
\begin{equation}
  \mathbf{n}! \equiv n_x! \cdot n_y! \cdot n_z!,
\end{equation}
which leads to a simple expression for the binomial coefficients of
two multi-indices entering Taylor expansions:
\begin{equation}
  \binom{\mathbf{n}}{\mathbf{k}} = \binom{n_x}{k_x}\binom{n_y}{k_y}\binom{n_z}{k_z}.
\end{equation}
When appearing as the index in a sum, a multi-index represents all
values that the triplet can take up to a given norm. For instance,
$\sum_{\mathbf{n}}^{p}$ indicates that the sum runs over all possible
multi-indices whose norm is $\leq p$.
