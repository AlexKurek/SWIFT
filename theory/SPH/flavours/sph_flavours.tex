\documentclass[fleqn, usenatbib, useAMS, a4paper]{mnras}
\usepackage{graphicx}
\usepackage{amsmath,paralist,xcolor,xspace,amssymb}
\usepackage{times}
\usepackage[super]{nth}

\newcommand{\swift}{{\sc swift}\xspace}
\newcommand{\gadget}{{\sc gadget}\xspace}
\newcommand{\dd}[2]{\frac{\partial #1}{\partial #2}}
\renewcommand{\vec}[1]{{\mathbf{#1}}}
\newcommand{\Wij}{\overline{\nabla_xW_{ij}}}

%\setlength{\mathindent}{0pt}

%opening
\title{SPH flavours in SWIFT}
\author{Matthieu Schaller}

\begin{document}

\maketitle

For vector quantities, we define $\vec{a}_{ij} \equiv (\vec{a}_i -
\vec{a}_j)$. Barred quantities ($\bar a_{ij}$) correspond to the
average over two particles of a quantity: $\bar a_{ij} \equiv
\frac{1}{2}(a_i + a_j)$. To simplify notations, we also define the
vector quantity $\Wij \equiv \frac{1}{2}\left(W(\vec{x}_{ij}, h_i) +
\nabla_x W(\vec{x}_{ij},h_j)\right)$.

\subsection{Minimal SPH}
\label{sec:sph:minimal}

This is the simplest fully-conservative version of SPH using the
internal energy $u$ as a thermal variable that can be written
down. Its implementation in \swift should be understood as a text-book
example and template for more advanced implementation. A full
derivation of the equations can be found in the review of
\cite{Price2012}. Our implementation follows their equations (27), (43),
(44), (45), (101), (103) and (104) with $\beta=3$ and $\alpha_u=0$. We
summarize the equations here. 

\subsubsection{Density and other fluid properties (\nth{1} neighbour loop)}

For a set of particles $i$ with positions $\vec{x}_i$ with velocities
$\vec{v}_i$, masses $m_i$, sthermal energy per unit mass $u_i$ and
smoothing length $h_i$, we compute the density for each particle:

\begin{equation}
  \rho_i \equiv \rho(\vec{x}_i) = \sum_j m_j W(\vec{x}_{ij}, h_i),
  \label{eq:sph:minimal:rho}
\end{equation}
and the derivative of its density with respect to $h$:

\begin{equation}
    \label{eq:sph:minimal:rho_dh}
  \rho_{\partial h_i} \equiv \dd{\rho}{h}(\vec{x}_i) = \sum_j m_j \dd{W}{h}(\vec{x}_{ij}
  , h_i).
\end{equation}
The gradient terms (``h-terms'') can then be computed from the density
and its derivative:

\begin{equation}
  f_i \equiv \left(1 + \frac{h_i}{3\rho_i}\rho_{\partial h_i}
  \right)^{-1}.
  \label{eq:sph:minimal:f_i}
\end{equation}
Using the pre-defined equation of state, the pressure $P_i$ and the sound
speed $c_i$ at the location of particle $i$ can now be computed from
$\rho_i$ and $u_i$:

\begin{align}
  P_i &= P_{\rm eos}(\rho_i, u_i),   \label{eq:sph:minimal:P}\\
  c_i &= c_{\rm eos}(\rho_i, u_i).   \label{eq:sph:minimal:c}
\end{align}

\subsubsection{Hydrodynamical accelerations (\nth{2} neighbour loop)}

We can then proceed with the second loop over
neighbours. The signal velocity $v_{{\rm sig},ij}$ between two particles is given by

\begin{align}
  \mu_{ij} &=
  \begin{cases}
  \frac{\vec{v}_{ij} \cdot \vec{x}_{ij}}{|\vec{x}_{ij}|}  & \rm{if}~
  \vec{v}_{ij} \cdot \vec{x}_{ij} < 0,\\
    0 &\rm{otherwise}, \\
  \end{cases}\nonumber\\
  v_{{\rm sig},ij} &= c_i + c_j - 3\mu_{ij}.   \label{eq:sph:minimal:v_sig}
\end{align}
We also use these two quantities for the calculation of the viscosity term:

\begin{equation}
\nu_{ij} = -\frac{1}{2}\frac{\alpha \mu_{ij} v_{{\rm
      sig},ij}}{\bar\rho_{ij}}
  \label{eq:sph:minimal:nu_ij}
\end{equation}
The fluid accelerations are then given by

\begin{align}
  \frac{d\vec{v}_i}{dt} = -\sum_j m_j &\left[\frac{f_iP_i}{\rho_i^2}
  \nabla_x W(\vec{x}_{ij}, h_i)   \nonumber
  +\frac{f_jP_j}{\rho_j^2}\nabla_x W(\vec{x}_{ij},h_j)\right. \\
  &+ \left. \bigg.\nu_{ij} \Wij \right], \label{eq:sph:minimal:dv_dt}
\end{align}
and the change in internal energy,

\begin{align}
  \frac{du_i}{dt} = \sum_j m_j &\left[\frac{f_iP_i}{\rho_i^2}  \vec{v}_{ij}
    \cdot \nabla_x W(\vec{x}_{ij}, h_i) \right. \label{eq:sph:minimal:du_dt}\\
    &+\left. \frac{1}{2}\nu_{ij}\vec{v}_{ij}\cdot\Big. \Wij\right], \nonumber
\end{align}
where in both cases the first line corresponds to the standard SPH
term and the second line to the viscuous accelerations.

We also compute an estimator of the change in smoothing length to be
used in the prediction step. This is an estimate of the local
divergence of the velocity field compatible with the accelerations
computed above:

\begin{equation}
  \frac{dh_i}{dt} = -\frac{1}{3}h_i \sum_j \frac{m_j}{\rho_j}
  \vec{v}_{ij}\cdot \nabla_x W(\vec{x}_{ij}, h_i).
  \label{eq:sph:minimal:dh_dt}
\end{equation}
and update the signal velocity of the particles:

\begin{equation}
  v_{{\rm sig},i} = \max_j \left( v_{{\rm sig},ij} \right).
  \label{eq:sph:minimal:v_sig_update}
\end{equation}
All the quantities required for time integration have now been obtained.

\subsubsection{Time integration}

For each particle, we compute a time-step given by the CFL condition:

\begin{equation}
  \Delta t = 2 C_{\rm CFL} \frac{H_i}{v_{{\rm sig},i}},
    \label{eq:sph:minimal:dt}
\end{equation}
where $C_{\rm CFL}$ is a free dimensionless parameter and $H_i = \gamma h_i$ is the
kernel support size. Particles can then be ``kicked'' forward in time:
\begin{align}
  \vec{v}_i &\rightarrow \vec{v}_i + \frac{d\vec{v}_i}{dt} \Delta t  \label{eq:sph:minimal:kick_v}\\
  u_i &\rightarrow u_i + \frac{du_i}{dt} \Delta t \label{eq:sph:minimal:kick_u}\\
  P_i &\rightarrow P_{\rm eos}\left(\rho_i, u_i\right) \label{eq:sph:minimal:kick_P}, \\
  c_i &\rightarrow c_{\rm eos}\left(\rho_i, u_i\right) \label{eq:sph:minimal:kick_c},
\end{align}
where we used the pre-defined equation of state to compute the new
value of the pressure and sound-speed.

\subsubsection{Particle properties prediction}

Inactive particles need to have their quantities predicted forward in
time in the ``drift'' operation. We update them as follows:

\begin{align}
  h_i &\rightarrow h_i \exp\left(\frac{1}{h_i} \frac{dh_i}{dt}
  \Delta t\right), \label{eq:sph:minimal:drift_h}\\
  \rho_i &\rightarrow \rho_i \exp\left(-\frac{3}{h_i} \frac{dh_i}{dt}
  \Delta t\right), \label{eq:sph:minimal:drift_rho} \\
  P_i &\rightarrow P_{\rm eos}\left(\rho_i, u_i + \frac{du_i}{dt} \Delta t\right), \label{eq:sph:minimal:drift_P}\\
  c_i &\rightarrow c_{\rm eos}\left(\rho_i, u_i + \frac{du_i}{dt}
  \Delta t\right) \label{eq:sph:minimal:drift_c},
\end{align}
where, as above, the last two updated quantities are obtained using
the pre-defined equation of state. Note that the thermal energy $u_i$
itself is \emph{not} updated.

\subsection{Gadget-2 SPH}
\label{sec:sph:gadget2}

This flavour of SPH is the one implemented in the \gadget-2 code
\citep{Springel2005}. The basic equations were derived by
\cite{Springel2002} and also includes a \cite{Balsara1995} switch for
the suppression of viscosity. The implementation here follows closely the
presentation of \cite{Springel2005}. Specifically, we use their equations (5), (7),
(8), (9), (10), (13), (14) and (17). We summarize them here for completeness.

\subsubsection{Density and other fluid properties (\nth{1} neighbour loop)}

For a set of particles $i$ with positions $\vec{x}_i$ with velocities
$\vec{v}_i$, masses $m_i$, entropic function per unit mass $A_i$ and
smoothing length $h_i$, we compute the density, derivative of the density with respect
to $h$ and the ``f-terms'' in a similar way to the minimal-SPH case
(Eq. \ref{eq:sph:minimal:rho}, \ref{eq:sph:minimal:rho_dh} and
\ref{eq:sph:minimal:f_i}). From these the pressure and sound-speed can
be computed using the predefined equation of state:

\begin{align}
  P_i &= P_{\rm eos}(\rho_i, A_i),   \label{eq:sph:gadget2:P}\\
  c_i &= c_{\rm eos}(\rho_i, A_i).   \label{eq:sph:gadget2:c}
\end{align}
We additionally compute the divergence and
curl of the velocity field using standard SPH expressions:

\begin{align}
  \nabla\cdot\vec{v}_i \equiv\nabla\cdot \vec{v}(\vec{x}_i) &= \frac{1}{\rho_i} \sum_j m_j
  \vec{v}_{ij}\cdot\nabla_x W(\vec{x}_{ij}, h_i) \label{eq:sph:gadget2:div_v},\\ 
    \nabla\times\vec{v}_i \equiv \nabla\times \vec{v}(\vec{x}_i) &= \frac{1}{\rho_i} \sum_j m_j
  \vec{v}_{ij}\times\nabla_x W(\vec{x}_{ij}, h_i) \label{eq:sph:gadget2:rot_v}.
\end{align}
These are used to construct the \cite{Balsara1995} switch $B_i$:

\begin{equation}
  B_i = \frac{|\nabla\cdot\vec{v}_i|}{|\nabla\cdot\vec{v}_i| +
    |\nabla\times\vec{v}_i| + 10^{-4}c_i / h_i}, \label{eq:sph:gadget2:balsara}
\end{equation}
where the last term in the denominator is added to prevent numerical instabilities.

\subsubsection{Hydrodynamical accelerations (\nth{2} neighbour loop)}

The accelerations are computed in a similar fashion to the minimal-SPH
case with the exception of the viscosity term which is now modified to
include the switch. Instead of Eq. \ref{eq:sph:minimal:nu_ij}, we get:

\begin{equation}
\nu_{ij} = -\frac{1}{2}\frac{\alpha \bar B_{ij} \mu_{ij} v_{{\rm sig},ij}}{\bar\rho_{ij}},
  \label{eq:sph:gadget2:nu_ij}  
\end{equation}
whilst equations \ref{eq:sph:minimal:v_sig},
\ref{eq:sph:minimal:dv_dt}, \ref{eq:sph:minimal:dh_dt} and
\ref{eq:sph:minimal:v_sig_update} remain unchanged. The only other
change is the equation of motion for the thermodynamic variable which
now has to be describing the evolution of the entropic function and
not the evolution of the thermal energy. It reads:

\begin{equation}
\frac{dA_i}{dt} = \frac{1}{2} \frac{\gamma-1}{\rho^{\gamma-1}} \sum_j
m_j \nu_{ij}\vec{v}_{ij}\cdot \Wij
\end{equation}

\subsubsection{Time integration}

The time-step condition is identical to the Minimal-SPH case
(Eq. \ref{eq:sph:minimal:dt}). The same applies to the integration
forward in time (Eq. \ref{eq:sph:minimal:kick_v} to
\ref{eq:sph:minimal:kick_c}) with the exception of the change in
internal energy (Eq. \ref{eq:sph:minimal:kick_u}) which gets replaced
by an integration for the the entropy:


\begin{align}
  \vec{v}_i &\rightarrow \vec{v}_i + \frac{d\vec{v}_i}{dt} \Delta t  \label{eq:sph:gadget2:kick_v}\\
  A_i &\rightarrow A_i + \frac{dA_i}{dt} \Delta t \label{eq:sph:gadget2:kick_A}\\
  P_i &\rightarrow P_{\rm eos}\left(\rho_i, A_i\right) \label{eq:sph:gadget2:kick_P}, \\
  c_i &\rightarrow c_{\rm eos}\left(\rho_i, A_i\right) \label{eq:sph:gadget2:kick_c},
\end{align}
where, once again, we made use of the equation of state relating
thermodynaimcal quantities.

\subsubsection{Particle properties prediction}

The prediction step is also identical to the Minimal-SPH case with the
entropic function replacing the thermal energy.

\begin{align}
  h_i &\rightarrow h_i \exp\left(\frac{1}{h_i} \frac{dh_i}{dt}
  \Delta t\right), \label{eq:sph:gadget2:drift_h}\\
  \rho_i &\rightarrow \rho_i \exp\left(-\frac{3}{h_i} \frac{dh_i}{dt}
  \Delta t\right), \label{eq:sph:gadget2:drift_rho} \\
  P_i &\rightarrow P_{\rm eos}\left(\rho_i, A_i + \frac{dA_i}{dt} \Delta t\right), \label{eq:sph:gadget2:drift_P}\\
  c_i &\rightarrow c_{\rm eos}\left(\rho_i, A_i + \frac{dA_i}{dt}
  \Delta t\right) \label{eq:sph:gadget2:drift_c},
\end{align}
where, as above, the last two updated quantities are obtained using
the pre-defined equation of state. Note that the entropic function $A_i$
itself is \emph{not} updated.



\subsection{Pressure-Entropy SPH}

\cite{Hopkins2013}
\label{sec:sph:pe}

\subsubsection{Hydrodynamical accelerations}

\subsubsection{Particle properties prediction}

\subsection{Pressure-Energy SPH}
\label{sec:sph:pu}

\cite{Hopkins2013}\\
to be done

\subsection{Anarchy SPH}
\label{sec:sph:anarchy}

to be done

\bibliographystyle{mnras}
\bibliography{./bibliography.bib}

\end{document}
