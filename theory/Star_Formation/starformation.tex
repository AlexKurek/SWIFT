\section{Star Formation in EAGLE}

In this section we will shortly explain how the star formation in EAGLE works.
The implemented star formation is based on the \citet{schaye2008}, instead of 
the constant density threshold used by \citet{schaye2008}, a metallicity 
dependent density threshold is used, following \citet{schaye2004}. An important 
property of the implemented star formation law is the explicit reproducability 
of the Kennicutt-Schmidt star formation law \citep{kennicutt1998}:
\begin{align}
 \dot{\Sigma}_\star &= A \left( \frac{\Sigma}{1 ~\text{M}_\odot ~\text{pc}^{-2}} \right)
\end{align}

\noindent In which $A$ is the normalization of the Kennicutt-Schmidt, $\dot{\Sigma}_\star$
is the surface density of newly formed stars, $\Sigma$ is the gas surface 
density and $n$ is the power law index. In the case of the star formation 
implementation of \citet{schaye2008}, the star formation law is given by 
a pressure law:

\begin{align}
\dot{m}_\star &= m_g A ( 1~\text{M}_\odot~\text{pc}^{-2})^{-n} \left( 
\frac{\gamma}{G} f_g P \right)^{(n-1)/2}.
\end{align}

\noindent In which $m_g$ is the gas particle mass, $\gamma$ is the ratio of specific heats,
$G$ is the gravitational constant, $f_g$ is the mass fraction of gas (unity in 
EAGLE), and $P$ is the total pressure of the gas particle. In this equation
$A$ and $n$ are directly constrainted from the observations of the Kennicutt-
Schmidt law so both variables do not require tuning. Further there are 
two constrains on the over density which should be $\Delta > 57.7$ (why this 
specific number? \citet{schaye2008} says $\Delta \approx 60$), and the
temperature of the gas should be atleast $T_\text{crit}<10^5 ~\text{K}$

Besides this it is required that there is an effective equation of state. 
Specifically we could take this to be equal to:
\begin{align}
 P &= P_\text{eos} (\rho) = P_\text{tot,c}\left( \frac{\rho_\text{g}}{\rho_\text{g,c}} \right)^{\gamma_\text{eff}}.
\end{align}
\noindent In which $\gamma_\text{eff}$ is the polytropic index. But the EAGLE 
code just uses the EOS of the gas?

\noindent Using this it is possible to calculate the propability that a gas particle is 
converted to a star particle:
\begin{align}
 \text{Prob.} = \text{min} \left( \frac{\dot{m}_\star \Delta t}{m_g}, 1 \right) 
 = \text{min} \left( A \left( 1 ~\text{M}_\odot ~\text{pc}^{-2} \right)^{-n} \left( \frac{\gamma}{G} f_g P_\text{tot} \right)^{(n-1)/2}, 1 \right).
\end{align}

\noindent In general we use $A=1.515 \cdot 10^{-4}~\text{M}_\odot ~\text{yr}^{-1} ~\text{kpc}^{-2}$ 
and $n=1.4$. In the case of high densities ($n_\text{H,thresh} > 10^3 ~\text{cm}^{-3}$),
the power law will be steaper and have a value of $n=2$ \citep{schaye2015}. This will also adjust
the normalization of the star formation law, both need to be equal at the 
pressure with a corresponding density. This means we have:
\begin{align}
\begin{split}
 A \left( 1 ~\text{M}_\odot ~\text{pc}^{-2} \right)^{-n} \left( \frac{\gamma}{G} f_g P_\text{tot} \right)^{(n-1)/2} \\
 = A_\text{high} \left( 1 ~\text{M}_\odot ~\text{pc}^{-2} \right)^{-n_\text{high}} \left( \frac{\gamma}{G} f_g P_\text{tot} \right)^{(n_\text{high}-1)/2}. 
\end{split}
\end{align}
\begin{align}
A_\text{high} = A \left( 1 ~\text{M}_\odot ~\text{pc}^{-2} \right)^{n_\text{high}-n} \left( \frac{\gamma}{G} f_g P_\text{tot}(\rho_{hd}) \right)^{(n-n_\text{high})/2}. 
\end{align}
In which $\rho_{hd}$ is the density at which both laws are equal.

This is differently from the EAGLE code ($f_g=1$) which uses:
\begin{align}
A_\text{high} = A \left( \frac{\gamma}{G} P_\text{tot} (\rho_{hd}) \right)^{(n-n_\text{high})/2} . 
\end{align}

Besides this we also use the metallicity dependent density threshold given by \citep{schaye2004}:
\begin{align}
n^*_\text{H} (Z) &= n_\text{H,norm} \left( \frac{Z}{Z_0} \right)^{n_z}.
\end{align}
In which $n_\text{H,norm}$ is the normalization of the metallicity dependent 
star formation law, $Z$ the metallicity, $Z_0$ the normalization metallicity,
and $n_Z$ the power law of the metallicity dependence on density. standard 
values we take for the EAGLE are $n_\text{H,norm} = 0.1 ~\text{cm}^{-3}$, 
$n_Z=-0.64$ and $Z_0 = 0.002$. Also we impose that the density threshold cannot
exceed the maximum value of $n_\text{H,max,norm}$ \citep{schaye2015}.

For the initial pressure determination the EAGLE code uses (Explanation needed):
\begin{align}
 P_\text{cgs} &= (\gamma -1) \frac{n_\text{EOS, norm} \cdot m_H}{X} T_{EOS,jeans} \cdot \frac{k_B}{1.22 \cdot (\gamma -1) m_H } \left( \frac{n_\text{highden}}{n_\text{norm,EOS}} \right)^{\gamma_\text{eff}}.
\end{align}

To determine the pressure for the star formation law the EAGLE code uses the 
physical pressure? Is this the effective EOS or the real EOS of the gas?

Compared to the EAGLE code we can calculate a fraction of the calculations already 
in the struct which are not depending on time, this may save some calculations. 

Besides this we also use the more extended temperature criteria proposed by
\citet{dallavecchia2012} that uses a temperature floor given by:
\begin{align}
 \log_{10} T < \log_{10} T_\text{eos} + 0.5.
\end{align}

\begin{table}
\begin{tabular}{l|l|l|l}
Variable & Parameter file name   & Default value & unit \\ \hline
$A$    & SchmidtLawCoeff\_MSUNpYRpKPC2   & $1.515\cdot10^{-4}$    & $M_\odot ~yr^{-1} ~kpc^{-2}$ \\
$n$  & SchmidtLawExponent                & $1.4$         & none  \\
$\gamma$  & gamma   & $\frac{5}{3}$ & none   \\
$G$  & No, in constants   & -  & -  \\
$f_g$ & fg   & $1.$    & none  \\
$n_{high}$   & SchmidtLawHighDensExponent  & $2.0$  & none  \\
$n_{H,thresh}$ & SchmidtLawHighDens\_thresh\_HpCM3 & $10^3$ & $cm^{-3}$ \\
$n_{H,norm}$ & thresh\_norm\_HpCM3 & $.1$ & $cm^{-3}$ \\
$Z_0$ & MetDep\_Z0 & $0.002$ & none \\
$n_Z$ & MetDep\_SFthresh\_Slope & $-0.64$ & none \\
$\Delta$ & thresh\_MinOverDens & $57.7$ & none \\
$T_{crit}$ & thresh\_temp & $10^5$ & $K$ \\
$n_{H,max,norm}$ & thresh\_max\_norm\_HpCM3 & 10.0 & $cm^{-3}$ 
\end{tabular}
\end{table}



