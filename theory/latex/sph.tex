\documentclass[a4paper,10pt]{article}
\usepackage[utf8]{inputenc}

%opening
\title{SPH equations}
\author{Matthieu}
\begin{document}

\maketitle

This document does not follow the GADGET notation.\\

\section{Particle definition}
Every particle contains the following information:

\begin{table}[h]
\centering
\begin{tabular}{|l|l|c|l|}
 Quantity & Type & Symbol & Unit \\
 \hline \hline
 Position & Primary & $\vec{x}$ & $[m]$ \\
 Velocity & Primary &$\vec{v}$ & $[m\cdot s^{-1}]$ \\
 Acceleration & Tertiary &$\vec{a}$ & $[m\cdot s^{-2}]$ \\
 Mass & Primary &$m$ & $[kg]$ \\
 Density & Secondary & $\rho$ & $[kg\cdot m^{-3}]$ \\
 Pressure & Secondary & $P$ & $[kg \cdot m^{-1}\cdot s^{-2}]$ \\
 Internal energy per unit mass & Primary & $u$ & $[m^2 \cdot s^{-2}]$ \\ 
 Energy derivative & Tertiary & $\frac{du}{dt}$ & $[ m^2 \cdot s^{-3}]$ \\
 Correction term & Secondary & $\Omega$ & $[-]$ \\
 Smoothing length & --- &$h$ & $[m]$ \\
 Smoothing length derivative & --- &$\frac{dh}{dt}$ & $[m\cdot s^{-1}]$ \\
 Time step & --- & $\Delta t$ & $[s]$ \\
\hline
\end{tabular} 
\end{table}

For optimisation purposes, any function of these qunatities could be stored. For instance, $1/h$ instead of $h$ or
$\frac{P}{\rho\Omega}$ instead of $\Omega$ may be options worth exploring.

\section{Kernel function}

In what follows, we will use $r_{ij} = \vec{x_i} - \vec{x_j}$ and the kernel function given by:

\begin{equation}
 W(\vec{x}, h) = \frac{1}{h^3}f\left(\frac{|\vec{x}|}{h}\right) 
\end{equation}

where $f(q)$ is a low-order polynomial. The simplest possible choice is the cubic-spline kernel which (in 3D) reads
\begin{eqnarray*}
 f(q) &=& \frac{1}{\pi}\left\lbrace \begin{array}{rcl}
                      \frac{1}{4}(2-q)^3 - (1-q)^3 & \mbox{if} & 0 \leq q < 1 \\
		      \frac{1}{4}(2-q)^3 & \mbox{if} & 1 \leq q < 2 \\
		      0 & \mbox{if} & q \geq 2 \\
                     \end{array}
 \right. \\
&=&\left\lbrace \begin{array}{rcl}
    0.31831 -0.477465 q^2+0.238732 q^3& \mbox{if} & 0 \leq q < 1 \\
   0.63662 -0.95493 q+0.477465 q^2-0.0795775 q^3  & \mbox{if} & 1 \leq q < 2 \\
		      0 & \mbox{if} & q \geq 2 \\
                     \end{array}
 \right.
\end{eqnarray*}

The constants here are NOT the constants used in GADGET as we are not following their convention of setting $h$ as the
cut-off value of $W$.\\
Notice that the kernel goes to $0$ when $r_{ij} = 2h$ in this case. The constant in front of $h$ depends on the kernel
chosen and to keep it general, we should insert a constant here and say that the interaction only takes place if
$r<\zeta h$ and keep $\zeta$ as a modifiable (compile time) constant. In other words, we can say that $W(x,h)$ is a
function that goes to $0$ if $x > \zeta h$. \\
Coming back to the simplest case, the derivatives of the kernel function are given by:

\begin{eqnarray*}
 \vec\nabla W(\vec{x},h) &=& \frac{1}{h^4}f'\left(\frac{|\vec{x}|}{h}\right) \frac{\vec{x}}{|\vec{x}|} \\
 \frac{\partial W(\vec{x},h)}{\partial h} &=&- \frac{1}{h^4}\left[3f\left(\frac{|\vec{x}|}{h}\right) + 
\frac{|\vec{x}|}{h}f'\left(\frac{|\vec{x}|}{h}\right)\right]
\end{eqnarray*}

with

\begin{eqnarray*}
  f'(q)&=& \frac{1}{\pi}\left\lbrace \begin{array}{rcl}
                      3 \left(1-q\right)^2-\frac{3}{4} \left(2-q\right)^2 & \mbox{if} &
0 \leq q < 1 \\
 		      -\frac{3}{4} \left(2-q\right)^2 & \mbox{if} & 1 \leq q < 2 \\
		      0 & \mbox{if} & q \geq 2 \\
                     \end{array}
 \right. \\
&=&\left\lbrace \begin{array}{rcl}
    -0.95493 q + 0.716197 q^2& \mbox{if} & 0 \leq q < 1 \\
   0.95493 -0.95493 q+0.238732 q^2  & \mbox{if} & 1 \leq q < 2 \\
		      0 & \mbox{if} & q \geq 2 \\
                     \end{array}
 \right.
\end{eqnarray*}

In summary, the SPH method uses a low order polynomial $f(q)$ which vanishes for any $q>\zeta$. Those two ``objects''
are linked together and are likely to be changed in different versions of the code. It would be great to be able to
change this without really touching the code. 
Having a compilation option somewhere which activates a given form of $f$ and changes the value of $\zeta$ accordingly
would be great.

\section{SPH equations}

In the first loop of the algorithm, the secondary quantities of particle $i$ are computed from the primary ones in the
following way:

\begin{eqnarray}
 \rho_i &=& \sum_j m_j W(r_{ij}, h_i)\\
 h_i &=& \eta \left(\frac{m_i}{\rho_i} \right)^{1/3}
\end{eqnarray}

where $\eta \approx 1.2$ is a constant. These two equations can be solved iterativelly using a Newton-Raphson or
bissection scheme. In practice, the loop is performed over all particles $j$ which are at a distance $r_{ij}<\zeta
h$ from the particle of interest. One has to iterate those two equations until their outcomes are stable. \\
To increase the convergence rate, one can use the derivative of the density with respect to the smoothing length in the
Newton iterations:

\begin{equation}
 \frac{\partial \rho}{\partial h} = \sum_j m_j \frac{\partial W(r_{ij},h_i)}{\partial h}
\end{equation}

This can also give a convergence criterion as this term must be $0$ when the right value oh $h$ has been found.
The derivative of the kernel function has to be computed anyway to obtain a value for the correction term $\Omega_i$.
This term is given by

\begin{equation}
  \Omega_i = 1 + \frac{h_i}{3\rho_i}\sum_b m_b\frac{\partial W(r_{ij},h_i)}{\partial h}
\end{equation}

Once those quantities have been obtained, the force estiamtion loop can be started.
First, the pressure has to be evaluated evaluated using the equation of state

\begin{equation}
 P_i = \rho_i u_i (\gamma - 1)
\end{equation}

where $\gamma$ is the polytropic index. Usually, $\gamma = \frac{5}{3}$.
The second loop is used to compute the accelerations (tertiary quantities). The exact expressions are

\begin{eqnarray}
 \vec{a} &=& - \sum_j m_j\left[\frac{P_i}{\Omega_i\rho_i^2}\vec{\nabla} W(r_{ij}, h_i) +
\frac{P_j}{\Omega_j\rho_j^2}\vec{\nabla}
W(r_{ij}, h_j) \right] \\
 \frac{du}{dt} &=& \frac{P_i}{\rho_i^2} \sum_j m_j (\vec{v_i}-\vec{v_j})\cdot\vec{\nabla} W(r_{ij}, h_i) \\
 \frac{dh}{dt} &=& \frac{h_i}{3}\sum_j \frac{m_j}{\rho_j} \left(\vec{v_j} - \vec{v_i} \right) \cdot\vec{\nabla}W(r_{ij},
h_i)
\end{eqnarray}

In practice the loop is here performed over all pairs of particles such that $r_{ij} < \zeta h_i$ or $r_{ij} < \zeta
h_j$. In general, the equations are more involved as they will contain terms to mimimc the effect of viscosity or
thermal conduction. These terms are pure functions of the properties of particles $i$ and $j$ and are thus very simple
to insert once the code is stabilized.\\

The time steps are computed using the speed of sound inside each ``kernel volume'' surrounding the particle:

\begin{equation}
 c_i = \sqrt{\frac{\gamma P_i}{\rho_i}} = \sqrt{\gamma (\gamma-1)u_i}
\end{equation}

The time step is then given by the Courant relation:

\begin{equation}
 \Delta t_i = C_{CFL} \frac{h_i}{c_i}
\end{equation}

where the CFL parameter usually takes a value between $0.2$ and $0.3$. The integration in time can then take place. The
leapfrog integrator is usually used as it behaves well when coupled to gravity. \\
In the case where only one global timestep is used for all particles, the minimal timestep of all particles is reduced
and used. \\

Notice that $h$ has to be recomputed through the iterative process
presented in the previous section at every timestep. The time
derivative of the smoothing length only give a rough estimate of its
change. It only provides a good guess for the Newton-Raphson (or
bissection) scheme.

\section{Conserved quantities}

The following quantities are exactly conserved by the code:

\begin{eqnarray}
\vec{P} &=&\sum_i m_i \vec{v_i}\\
\vec{L} &=& \sum_i m_i \vec{x_i} \times \vec{v_i}\\ 
E &=&\sum_i m_i\left(\frac{1}{2}|\vec{v_i}|^2+u_i\right)
\end{eqnarray}


\end{document}
